\documentclass[a4paper]{article}
\usepackage[italian]{babel}
\usepackage[utf8]{inputenc}
\usepackage{float}
\usepackage{hyperref}
\usepackage{glossaries}

\title{Relazione di progetto di Virtualizzazione e integrazione di sistemi\\
\textbf{Server NFS e volumi di container}}
\date{\today}
\author{Luca Casadei\\Matricola: 0001069237}
\makeglossaries

\loadglsentries{defns}
\setacronymstyle{short-long}

\begin{document}
\maketitle

\tableofcontents

\section{Creazione delle macchine virtuali}
In questa sezione verrà indicata la modalità utilizzata per la creazione delle macchine virtuali, 
in particolar modo verrà descritto il modo usato per metterle in comunicazione.
\subsection{Ambiente di virtualizzazione}
Come ambiente di virtualizzazione è stato scelto \textit{Proxmox}, un software debian-based open source che permette di gestire macchine virtuali basato
sull'infrastruttura \gls{KVM} che fornisce già un repository di immagini \gls{LXC}.


\printglossaries
\end{document}